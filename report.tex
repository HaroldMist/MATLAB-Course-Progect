\documentclass[UTF8]{article}
\usepackage{ctex}
\usepackage{abstract}
\usepackage{algorithm}
\usepackage{algorithmic}
\usepackage{amsmath}
\usepackage{amssymb}
\usepackage{amsthm}
\usepackage{array}
\usepackage{bm}
\usepackage{caption}
\usepackage{cite}
% \usepackage{CJK}
\usepackage{enumitem}
\usepackage{float}
\usepackage{graphicx}
\usepackage{listings}
\usepackage{multirow}
\usepackage{subfigure}


\renewcommand{\abstractname}{}
\newcommand{\upcite}[1]{\textsuperscript{\textsuperscript{\cite{#1}}}}


\usepackage[a4paper,left=25mm,right=25mm,top=20mm,bottom=20mm]{geometry}

\lstset{
numbers=left,                           % 在左侧显示行号
numbersep=1em,
showstringspaces=false                % 不显示字符串中的空格
}

\begin{document}


    \tableofcontents
    \newpage


    \section{题目}
    
    \subsection{森林救火问题}

        森林失火了,消防站接到报警后派多少消防队员前去救火呢?派的队员越多,森林的损失越小,但是救援的开支会越大,所以需要综合考虑森林损失费和救援费与消防队员人数之间的关系,以总费用最小来决定派出队员的数目.

    \subsubsection{问题分析}
        
        损失费通常正比于森林烧毁的面积,而烧毁面积与失火、灭火(指火被扑灭)的时间有关,灭火时间又取决于消防队员数目,队员越多灭火越快.救援费除与消防队员人数有关外,也与灭火时间长短有关.记失火时刻为$t=0$,开始救火时刻为$t=t_1$,灭火时刻为$t=t_2$.设在时刻$t$森林烧毁面积为$B(t)$ ,则造成损失的森林烧毁面积为$B(t_2)$.建模要对函数$B(t)$的形式作出合理的简单假设.
    
        研究$\frac{dB}{dt}$比$B(t)$更为直接和方便.$\frac{dB}{dt}$是单位时间烧毁面积,表示火势蔓延的程度.在消防队员到达之前,即$0 \leqslant t \leqslant t_1$火势越来越大,即$\frac{dB}{dt}$随$t$的增加而增加;开始救火以后,即$t_1 \leqslant t \leqslant t_2$.如果消防队员救火能力足够强,火势会越来越小,即$\frac{dB}{dt}$应减小,并且当$t=t_2$时$\frac{dB}{dt}=0$.

        救援费可分为两部分;一部分是灭火器材的消耗及消防队员的薪金等,与队员人数及灭火所用的时间均有关,另一部分是运送队员和器材等一次性支出,只与队员人数有关.
    
    \subsubsection{模型假设}
        
        需要对烧毁森林的损失费、救援费及火势蔓延程度$\frac{dB}{dt}$的形式作出假设.

        \begin{itemize}
            \item[]
            1.损失费与森林烧毁面积$B(t_2)$成正比,比例系数$c_1$,$c_1$即烧毁单位面积的损失费.

            \item[]
            2.从失火到开始救火这段时间($0 \leqslant t \leqslant t_1$)内,火势蔓延程度$\frac{dB}{dt}$与时间$t$成正比,比例系数$\beta$称火势蔓延速度.

            \item[]
            3.派出消防队员$x$名,开始救火以后($t \geqslant t_1$)火势蔓延速度降为$\beta - \lambda x$,其中$\lambda$可视为每个队员的平均灭火速度.显然应有 $\beta < \lambda x$

            \item[]
            4.每个消防队员单位时间的费用为$c_2$,于是每个队员的救火费用是$c_2(t_2-t_1)$;每个队员的一次性支出是$c_3$.
        \end{itemize}


        第2条假设可作如下解释:火势以失火点为中心,以均匀速度向四周呈圆形蔓延.所以蔓延的半径$r$与时间$t$成正比,又因为烧毁面积$B$与$r^2$成正比,故$B$与$t^2$成正比,从而$\frac{dB}{dt}$与$t$成正比.

    \subsubsection{仿真要求}
        
        系统输入为派出消防队员$x$名,每个队员的平均灭火速度$\lambda$,火势蔓延速度$\beta$,开始救火时刻$t_1$,烧毁单位面积的损失费$c_1$,每个消防队员单位时间的费用$c_2$,每个队员的一次性支出$c_3$.系统输出为损失费和救援费以及总费用,灭火时刻$t_2$,森林烧毁面积$B$.要求有输入、输出界面及仿真过程.
        

    % \begin{figure}[H]
    %     \centering
    %     \includegraphics[width=0.8\textwidth]{1因果图.png}
    %     \caption{因果图}
    %     \label{1}
    % \end{figure}


    \section{建模过程}
        \subsection{模型建立}
        由题目中的模型假设部分可知,系统仿真主要是需要求解以下关于烧毁面积的公式,其中$t_2$为未知量,其余均为用户输入的已知量。
        $$\int_{0}^{t_1}\beta t \,dt + \int_{t_1}^{t_2}(\beta - \lambda x)t \,dt = 0 $$
        
        由于$t_1$时刻后,派出的消防员使得火势蔓延速度降为$\beta - \lambda x < 0$,所以面积应在开始灭火后不断减小,在$t_1$时刻的面积即为烧毁的最大面积。
        
        在求解出$t_2$后,即可计算出救援支出以及森林烧毁造成的损失,进一步计算出总的经济损失。
        具体公式如下:
        
        森林烧毁造成的损失:
        $$c_1 B = c_1 \int_{0}^{t_1}\beta t \,dt$$
        
        救援支出:
        $$[c_2(t_2 - t_1) + c_3]x$$
        
        总经济损失:
        $$c_1 \int_{0}^{t_1}\beta t \,dt + [c_2(t_2 - t_1) + c_3]x$$


        \subsection{模型求解}
        先通过MATLAB中的int命令计算第一部分$\int_{0}^{t_1}\beta t \,dt$的积分值,

        \subsection{关键难点}


    \section{运行指南}


    \section{运行实例}
    



    % \begin{lstlisting}
    % * COVID-19 MODEL
    % SPEC DT=1/LENGTH=30/PRTPER=0/PLTPER=1
    % L    L1.K = L1.J + (DT)*(R1.JK-R2.JK)
    % N    L1 = 0
    % NOTE
    % R    R1.KL = R1.JK*(1-A1*C1+A2*C2)
    % N    R1 = 7.5
    % C    C1 = 0.5
    % C    C2 = 0.5
    % NOTE
    % A    A1 = L2.K/L2.J*C3
    % N    A1 = 0
    % A    A2 = L1.K/L1.J+L2.K/L2.J
    % N    A2 = 0
    % C    C3 = 1

    % \end{lstlisting}
        
    
    % \bibliographystyle{plain}
    % \bibliography{1ref}

\end{document}